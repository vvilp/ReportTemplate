\section{Test Chinese}
测试中文, 一二三四五六七
\section{Section Headings}
We \citep{Ancey1996} BBB \citep{RR73} in this section how to obtain headings
for the various sections and subsections of our
document.
\subsection{Headings in the ‘article’ Document Style}
In the ‘article’ style, the document may be divided up
into sections, subsections and subsubsections, and each
can be given a title, printed in a boldface font,
simply by issuing the appropriate command.

The foundations\footnote{Inside minipage} of the rigorous study of \textit{analysis}
were laid in the nineteenth century, notably by the
mathematicians Cauchy and Weierstrass. Central to the
study of this subject are the formal definitions of
\textit{limits} and \textit{continuity}.

Let $D$ be a subset of $\bf R$ and let
$f \colon D \to \textbf{R}$ be a real-valued function on
$D$. The function $f$ is said to be \textit{continuous} on
$D$ if, for all $\epsilon > 0$ and for all $x \in D$,
there exists some $\delta > 0$ (which may depend on $x$)
such that if $y \in D$ satisfies
\[ |y - x| < \delta \]
then
\[ |f(y) - f(x)| < \epsilon. \]

One may readily verify that if $f$ and $g$ are continuous
functions on $D$ then the functions $f+g$, $f-g$ and
$f.g$ are continuous. If in addition $g$ is everywhere
non-zero then $f/g$ is continuous.

\section{algorithm}

\begin{algorithm}
\DontPrintSemicolon
\KwData{$G=(X,U)$ such that $G^{tc}$ is an order.}
\KwResult{$G’=(X,V)$ with $V\subseteq U$ such that $G’^{tc}$ is an
interval order.}
\Begin{
$V \longleftarrow U$\;
$S \longleftarrow \emptyset$\;
\For{$x\in X$}{
$NbSuccInS(x) \longleftarrow 0$\;
$NbPredInMin(x) \longleftarrow 0$\;
$NbPredNotInMin(x) \longleftarrow |ImPred(x)|$\;
}
\For{$x \in X$}{
\If{$NbPredInMin(x) = 0$ {\bf and} $NbPredNotInMin(x) = 0$}{
$AppendToMin(x)$}
}
\nl\While{$S \neq \emptyset$}{\label{InRes1}
\nlset{REM} remove $x$ from the list of $T$ of maximal index\;\label{InResR}
\lnl{InRes2}\While{$|S \cap ImSucc(x)| \neq |S|$}{
\For{$ y \in S-ImSucc(x)$}{
\{ remove from $V$ all the arcs $zy$ : \}\;
\For{$z \in ImPred(y) \cap Min$}{
remove the arc $zy$ from $V$\;
$NbSuccInS(z) \longleftarrow NbSuccInS(z) - 1$\;
move $z$ in $T$ to the list preceding its present list\;
\{i.e. If $z \in T[k]$, move $z$ from $T[k]$ to
$T[k-1]$\}\;
}
$NbPredInMin(y) \longleftarrow 0$\;
$NbPredNotInMin(y) \longleftarrow 0$\;
$S \longleftarrow S - \{y\}$\;
$AppendToMin(y)$\;
}
}
$RemoveFromMin(x)$\;
}
}
\caption{IntervalRestriction\label{IR}}
\end{algorithm}

\section{math}

\subsection{inline equation}
In physics, the mass-energy equivalence is stated
by the equation $E=mc^2$, discovered in 1905 by Albert Einstein.

\subsection{independent}
The mass-energy equivalence is described by the famous equation
$$E=mc^2$$
discovered in 1905 by Albert Einstein.
In natural units ($c$ = 1), the formula expresses the identity

\begin{equation}
E=m
\end{equation}

\begin{equation}
\int \oint \sum \prod \subset \supset \subseteq \supseteq \alpha \beta \gamma \rho \sigma \delta \epsilon
\end{equation}

Let $\mathbf{u}$,$\mathbf{v}$ and $\mathbf{w}$ be three
vectors in ${\mathbf R}^3$. The volume~$V$ of the
parallelepiped with corners at the points
$\mathbf{0}$, $\mathbf{u}$, $\mathbf{v}$,
$\mathbf{w}$, $\mathbf{u}+\mathbf{v}$,
$\mathbf{u}+\mathbf{w}$, $\mathbf{v}+\mathbf{w}$
and $\mathbf{u}+\mathbf{v}+\mathbf{w}$
is given by the formula
\[ V = (\mathbf{u} \times \mathbf{v}) \cdot \mathbf{w}.\]

\begin{equation}
f(x,y,z) = 3y^2 z \left( 3 + \frac{7x+5}{1 + y^2} \right)
\end{equation}


In non-relativistic wave mechanics, the wave function
     $\psi(\mathbf{r},t)$ of a particle satisfies the
     \textit{Schr\"{o}dinger Wave Equation}
     \[ i\hbar\frac{\partial \psi}{\partial t}
       = \frac{-\hbar^2}{2m} \left(
         \frac{\partial^2}{\partial x^2}
         + \frac{\partial^2}{\partial y^2}
         + \frac{\partial^2}{\partial z^2}
       \right) \psi + V \psi.\]
     It is customary to normalize the wave equation by
     demanding that
     \[ \int \!\!\! \int \!\!\! \int_{\textbf{R}^3}
           \left| \psi(\mathbf{r},0) \right|^2\,dx\,dy\,dz = 1.\]
     A simple calculation using the Schr\"{o}dinger wave
     equation shows that
     \[ \frac{d}{dt} \int \!\!\! \int \!\!\! \int_{\textbf{R}^3}
           \left| \psi(\mathbf{r},t) \right|^2\,dx\,dy\,dz = 0,\]
     and hence
     \[ \int \!\!\! \int \!\!\! \int_{\textbf{R}^3}
           \left| \psi(\mathbf{r},t) \right|^2\,dx\,dy\,dz = 1\]
     for all times~$t$. If we normalize the wave function in this
34
4 4.1
way then, for any (measurable) subset~$V$ of $\textbf{R}^3$
and time~$t$,
\[ \int \!\!\! \int \!\!\! \int_V
      \left| \psi(\mathbf{r},t) \right|^2\,dx\,dy\,dz\]
represents the probability that the particle is to be found
within the region~$V$ at time~$t$.

\section{list}

\begin{enumerate}
\item
$d(x,y) \geq 0$ for all points $x$ and $y$ of $X$;
\item
$d(x,y) = d(y,x)$ for all points $x$ and $y$ of $X$;
\item
$d(x,z) \leq d(x,y) + d(y,z)$ for all points $x$, $y$
and $z$ of $X$;
\item
$d(x,y) = 0$ if and only if the points $x$ and $y$
coincide.
\end{enumerate}

\begin{itemize}
\item
$d(x,y) \geq 0$ for all points $x$ and $y$ of $X$;
\item
$d(x,y) = d(y,x)$ for all points $x$ and $y$ of $X$;
\item
$d(x,z) \leq d(x,y) + d(y,z)$ for all points $x$, $y$
and $z$ of $X$;
\item

$d(x,y) = 0$ if and only if the points $x$ and $y$
coincide.
\end{itemize}

\begin{description}
\item[test1] AAAAAA
\item[test2] AAAAAA
\end{description}

\section{figure}

\begin{figure}[htbp!]
\centering
\includegraphics[width=0.8\textwidth]{Figs/1}
\caption[Minion]{This is just a long figure caption for the minion in Despicable Me from Pixar}
\label{fig:minion}
\end{figure}

\section{box}

\begin{framed}
This is an easy way to box text within a document!
\end{framed}

\section{box and code}

% \begin{framed}
% \inputminted[fontsize=\footnotesize]{c}{Code/main.c}
% \end{framed}
