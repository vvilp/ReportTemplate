% ************************** Thesis Abstract *****************************
\begin{abstract}
Next Generation Sequencing (NGS) technologies have led to an explosion in genetic and genomic datasets which has posed a significant challenge in existing sequence processing and searching methods, and has motivated new approaches for sequence processing including adaptation of methods from other domains such as information retrieval and deep learning. Measurement of sequence similarity is an important topic in bioinformatics. At present, such process is commonly conducted by BLAST and CLUSTAL, the alignment based similarity determination tools. However, alignment-based approaches may not adapt well on increasing size of datasets as well as on mutation and rearrangement of large gene fragments.

To tackle this problems of genetic data processing and searching, I have reviewed several conventional processing approaches (alignment methods) in bioinformatics, and compare to the alignment free method using locality-sensitive hashing methods. To tackle the problem of pattern and representation learning in bioinformatics, deep learning is a notable topic in such field which may provide us another direction for data processing in bioinformatics. 
\end{abstract}
